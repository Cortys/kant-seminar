% chktex-file 21
% chktex-file 46
\documentclass{llncs}

\PassOptionsToPackage{utf8}{inputenc}
\usepackage{inputenc}
\usepackage[ngerman]{babel}
\usepackage[compact]{titlesec}

% \usepackage[sc]{mathpazo}
\usepackage{geometry}
\geometry{
  a4paper,         % or letterpaper
  textwidth=17cm,  % llncs has 12.2cm
  textheight=25cm, % llncs has 19.3cm
  heightrounded,   % integer number of lines
  hratio=1:1,      % horizontally centered
  vratio=1:1,      % vertically centered
}
\renewcommand{\baselinestretch}{1.15}
\usepackage[parfill]{parskip}

\usepackage{amssymb}
\renewcommand{\labelitemi}{\raisebox{0.09cm}{\tiny\(\blacktriangleright\)}}
\renewcommand{\labelitemii}{\raisebox{0.09cm}{\tiny\(\blacktriangleright\)}}

\usepackage{graphicx}
\usepackage{enumitem}

\usepackage{hyperref}
\hypersetup{% setup the hyperref-package options
    pdftitle={Kant: Grundlegung zur Metaphysik der Sitten},    %   - title (PDF meta)
    pdfsubject={Thesenpapier},%   - subject (PDF meta)
    pdfauthor={Clemens Damke},    %   - author (PDF meta)
    plainpages=false,           %   -
    colorlinks=false,           %   - colorize links?
    pdfborder={0 0 0},          %   -
    breaklinks=true,            %   - allow line break inside links
    bookmarksnumbered=true,     %
    bookmarksopen=true          %
}

\RequirePackage{lmodern}	% font set: Latin Modern
		\RequirePackage{XCharter}	% font set: Charter
		%\RequirePackage{fourier}	% font set: (basically improved utopia)

		\RequirePackage[sfdefault, light]{roboto}
		\renewcommand*\familydefault{\rmdefault}

\begin{document}
\pagestyle{plain}

\title{Thesenpapier – Kant: Grundlegung zur Metaphysik der Sitten}
\subtitle{Abschnitt 2: Übergang zur Metaphysik der Sitten (S.\@ 58--69)}
\author{Clemens Damke (Matr.-Nr. 7011488)}
\date{27.05.2019}
\maketitle

\subsection*{Absatz 45 (58f.): Metaphysik der Sitten als Erklärungsgrundlage des kategorischen Imperativs}

\begin{itemize}
    \item \textbf{These:} Um die notwendige Gültigkeit des kategorischen Imperativs zu erkennen, wird eine Metaphysik der Sitten benötigt.
    \item \textbf{Begründung:} Da objektiv-praktische Gesetze gesucht werden, muss das Verhältnis des Willens zu sich selbst untersucht werden, nicht die Gründe des Handelns (präskriptive, nicht deskriptive Gesetze).
		Empirisch begründete Gesetze fallen weg, weil die Realität für das, was sein soll, unerheblich ist. Die Untersuchung des kategorischen Imperativs ist somit meta\-physisch.
\end{itemize}

\subsection*{Absatz 46 (59): Unterscheidung von subjektiven und objektiven Zwecken}

\begin{itemize}
    \item \textbf{Definitionen:}
		\begin{itemize}
			\item \textit{Triebfeder \& Bewegungsgrund}: Der subjektive bzw.\@ objektive Grund eines Begehrens.
			\item \textit{Mittel}: Der Grund der Möglichkeit einer Handlung.
			\item \textit{Objektiver Zweck}: Objektiver Grund der Selbstbestimmung des Willens. Beruht auf Bewegungsgründen, nicht auf Triebfedern, und gilt für alle vernünftigen Wesen.
			\item \textit{Subjektive/relative Zwecke}: Auf Triebfedern beruhender Grund der Selbstbestimmung des Willens. Nicht allgemein gültig und somit relativ.
		\end{itemize}
    \item \textbf{Konsequenz:} Relative Zwecke können nur hypothetische Imperative begründen, da sie keine Allgemein\-gültigkeit haben.
\end{itemize}

\subsection*{Absatz 47 (59): Einführung des Zwecks an sich}

\textbf{These:} Ausschließlich etwas, dessen Zweck das eigene Dasein ist, kann kategorische Imperative begründen.

\subsection*{Absatz 48 (59f.): Der Mensch als Zweck an sich}

\begin{itemize}
	\item \textbf{These:} Der Mensch und jedes andere vernünftige Wesen existiert als Zweck an sich selbst.
	\item \textbf{Konsequenz:} Vernünftige Wesen, d.~h.\@ Personen, sind objektive Zwecke. Sie dürfen nicht als Mittel gebraucht werden, sondern verdienen Achtung.
\end{itemize}

\subsection*{Absatz 49 (60f.): Formulierung des praktischen Imperativs}

\begin{itemize}
	\item \textbf{These:} Da vernünftige Wesen objektive Zwecke sind, begründen sie als oberster praktischer Grund den kate\-gorischen Imperativ.
	\item \textbf{Konsequenz:} Hieraus lässt sich der praktische Imperativ ableiten: \textit{Handle so, dass du die Menschheit [\ldots] jederzeit zugleich als Zweck, niemals bloß als Mittel brauchst}.
\end{itemize}

\subsection*{Absatz 51 (61): Praktischer Imperativ zum Selbstmord}

\textbf{Beispiel:} Ein Selbstmörder bedient sich seiner eigenen Person als Mittel zur Flucht aus einem beschwerlichen Zustand. $\Rightarrow$ \textit{Nicht mit dem praktischen Imperativ kompatibel.}

\subsection*{Absatz 52 (61f.): Praktischer Imperativ zum lügenhaften Versprechen}

\textbf{Beispiel:} Jemand, der Versprechen gibt, ohne sie halten zu wollen, bedient sich einer Person als Mittel, denn kein vernünftiges Wesen sieht seinen Zweck darin betrogen zu werden. $\Rightarrow$ \textit{Nicht mit dem praktischen Imperativ kompatibel.}

\subsection*{Absatz 53 (62): Praktischer Imperativ zur Beförderung der eigenen Fähigkeiten}

\textbf{Beispiel:} Die eigenen Fähigkeiten und Möglichkeiten nicht zu nutzen, ist zwar mit dem Zweck des Erhalts der Menschheit kompatibel, es befördert diesen Zweck jedoch auch nicht. $\Rightarrow$ \textit{Unvollständige Vorstellung des Menschen als Zweck.}

\subsection*{Absatz 54 (62f.): Praktischer Imperativ zur Beförderung Anderer}

\textbf{Beispiel:} Jemand, der zur Glückseligkeit anderer Menschen nichts beträgt, schadet diesen Menschen zwar nicht, er befördert deren Zwecke jedoch auch nicht. $\Rightarrow$ \textit{Unvollständige Vorstellung des Menschen als Zweck.}

\subsection*{Absatz 55 (63): Einführung des allgemein gesetzgebenden Willens}

\textbf{These:} Da der objektive Grund des praktischen Gesetzes in der Form und Allgemeinheit liegt und der subjektive im Zweck, d.~h.\@ im vernünftigen Wesen selbst, muss der Wille vernünftiger Wesen als allgemein gesetzgebend verstanden werden.

\subsection*{Absatz 56 (63f.): Der Wille als eigener Gesetzgeber}

\textbf{These:} Aus dem vorigen Absatz folgt, dass der Wille sich selbst als Gesetzgeber unterworfen ist. Seiner eigenen Gesetzgebung widersprechende Maximen hat er zu verwerfen.

\subsection*{Absatz 57 (64): Der allgemein-gesetzgebende Wille als Bestimmungsmerkmal des kat.\@ Imperativs}

\textbf{These:} Die Existenz praktischer kategorischer Sätze wurde bis hierhin nicht bewiesen.
Die Idee des allgemein-gesetzgebenden Willens kann jedoch verwendet werden, um den kategorischen Imperativ im Gegensatz zum hypothetischen zu bestimmen.

\subsection*{Absatz 58 (64): Ausschluss von Interessen im allgemein gesetzgebenden Willen}

\textbf{Begründung:} Die vorherige These wird dadurch begründet, dass ein allgemein-gesetzgebender Wille, der von Interessen abhängt, eines Gesetzes bedürfte welches die Gesetze mit seinen Interessen in Einklang bringt.
Somit wäre der Wille allerdings nicht mehr gesetzgebend (da er ja einem Gesetz folgen müsste), was einen inneren Widerspruch produzieren würde.

\subsection*{Absatz 59 (64f.): Die alleinige Unbedingtheit des gesetzgebenden Willens}

\textbf{These:} Der kategorische Imperativ muss, sofern er existiert, unbedingt fordern, dass ein ihm gemäßer Wille sich selbst als allgemein gesetzgebend versteht.
Ein solcher Wille ist nämlich, wie zuvor erörtert, frei von Inter\-essen.

\subsection*{Absatz 60 (65f.): Das Prinzip der Autonomie des Willens}

\begin{itemize}
	\item \textbf{Problem:} In vorigen Versuchen (d.~h.\@ vor Kant) das Prinzip der Sittlichkeit auszumachen, wurde, laut Kant, immer versucht ein externes Interesse daran zu identifizieren.
		So blieb die Pflicht unerkannt und es wurden keine allgemeinen moralischen Gebote gefunden.
	\item \textbf{Definitionen:} Der in den vorigen Absätzen beschriebene Grundsatz des gesetzgebenden Willens, wird \textit{Prinzip der Autonomie des Willens} genannt.
		Alle anderen Grundsätze, werden als \textit{Heteronomie} bezeichnet.
\end{itemize}

\subsection*{Absatz 61 (66): Einführung des Begriffs des \textit{Reichs der Zwecke}}

\textbf{These:} Das Prinzip der Autonomie des Willens führt zum \textit{fruchtbaren Begriff} des \textit{Reichs der Zwecke}.

\subsection*{Absatz 62, 63 \& 64 (66): Definition des \textit{Reichs der Zwecke}}

\textbf{Definition:} Das \textit{Reich der Zwecke} bezeichnet die systematische Verbindung aller vernünftiger Wesen über ihre gemeinschaftlichen objektiven Gesetze.
Das Reich der Zwecke umfasst:
\begin{itemize}
	\item Jedes vernünftige Wesen, da es zugleich Zweck an sich ist. Vernünftige Wesen, die nicht dem Willen eines anderen unterworfen sind, sind Oberhäupter im Reich der Zwecke.
	\item Die Privatzwecke, die zu den durch vernünftige Wesen gegebenen Gesetzen gehören.
\end{itemize}

\subsection*{Absatz 65 (67): Behauptung als Oberhaupt im \textit{Reich der Zwecke}}

\textbf{These:} Um als Wesen im Reich der Zwecke nicht nur Glied, sondern Oberhaupt zu sein, muss es seinen Willen unabhängig von anderen, im Rahmen des eigenen Vermögens, umsetzen.
Die Maxime des Willens ist hierbei unerheblich.

\subsection*{Absatz 66 (67): Beziehung zwischen Moralität und Pflicht}

\textbf{These:} Moralität besteht in der folgenden Beziehung zwischen Handlung und Gesetzgebung:
Handelt ein Wesen nicht bereits durch seine Natur nach einem Gesetz, so handelt es durch praktische Nötigung, d.~h.\@ Pflicht.

\subsection*{Absatz 67 (67): Einführung des Begriffs der Würde}

\textbf{These:} Die Pflicht beruht nicht auf Gefühlen, Antrieben oder Neigungen, sondern der Idee der Würde eines vernünftigen Wesens, welches selbst gegebenen Gesetzen gehorcht.

\subsection*{Absatz 68 (68): Gegenüberstellung von Preis und Würde}

\textbf{These:} Jeder Zweck im Reich der Zwecke hat entweder einen Preis oder Würde.
Dinge mit Preis lassen sich durch andere Dinge substituieren, Dinge mit Würde sind nicht ersetzbar.

\subsection*{Absatz 69 (68): Gegenüberstellung von Marktpreis, Affektionspreis und innerem Wert}

\textbf{These:} Dinge mit Preis lassen sich in zwei Klassen einordnen:
\begin{itemize}
	\item Dinge mit Bezug auf allgemeine menschliche Bedürfnisse haben einen \textit{Marktpreis} (z.~B.\@ Geschicklichkeit).
	\item Dinge zur Befriedigung des Gemüts haben einen \textit{Affektionspreis} (z.~B.\@ Witz).
\end{itemize}
Dinge mit Würde, d.~h.\@ einem inneren Wert, hingegen machen die Bedingung für etwas aus Zweck an sich zu sein (z.~B.\@ Treue).

\subsection*{Absatz 70 (68f.): Moralität als Bedingung für den Zweck an sich}

\begin{itemize}
	\item \textbf{These:} Moralität ist Voraussetzung für ein Wesen Zweck an sich zu sein.
	\item \textbf{Begründung:} Nur durch Moralität kann ein Wesen gesetzgebendes Glied im Reich der Zwecke sein.
	\item \textbf{Konsequenz:} Nur die Sittlichkeit und die Menschheit, sofern sie zur Sittlichkeit fähig ist, hat Würde.
		Diese Würde ist unabhängig von den Wirkungen, die aus ihr gemäßen Handlungen erwachsen.
\end{itemize}

\subsection*{Absatz 71 (69): Autonomie als Grund der Würde}

\textbf{These:} Die Tugend verschafft einem Wesen Anteil an der allgemeinen Gesetzgebung.
Es schafft dadurch also seine eigenen Gesetze, was ihm Freiheit, bzw.\@ Autonomie, verschafft.
Hierin besteht der Grund der Würde.

\end{document}
