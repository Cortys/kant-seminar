% chktex-file 21
% chktex-file 46
\documentclass{llncs}

\PassOptionsToPackage{utf8}{inputenc}
\usepackage{inputenc}
\usepackage[ngerman]{babel}

% \usepackage[sc]{mathpazo}
\usepackage{geometry}
\geometry{
  a4paper,         % or letterpaper
  textwidth=15cm,  % llncs has 12.2cm
  textheight=23cm, % llncs has 19.3cm
  heightrounded,   % integer number of lines
  hratio=1:1,      % horizontally centered
  vratio=1:1,      % vertically centered
}
\renewcommand{\baselinestretch}{1.15}
\usepackage[parfill]{parskip}

\usepackage{amssymb}
\renewcommand{\labelitemi}{\raisebox{0.09cm}{\tiny\(\blacktriangleright\)}}

\usepackage{graphicx}
\usepackage{enumitem}

\usepackage{hyperref}
\hypersetup{% setup the hyperref-package options
    pdftitle={Kant: Grundlegung zur Metaphysik der Sitten},    %   - title (PDF meta)
    pdfsubject={Thesenpapier},%   - subject (PDF meta)
    pdfauthor={Clemens Damke},    %   - author (PDF meta)
    plainpages=false,           %   -
    colorlinks=false,           %   - colorize links?
    pdfborder={0 0 0},          %   -
    breaklinks=true,            %   - allow line break inside links
    bookmarksnumbered=true,     %
    bookmarksopen=true          %
}

\RequirePackage{lmodern}	% font set: Latin Modern
		\RequirePackage{XCharter}	% font set: Charter
		%\RequirePackage{fourier}	% font set: (basically improved utopia)

		\RequirePackage[sfdefault, light]{roboto}
		\renewcommand*\familydefault{\rmdefault}

\begin{document}
\pagestyle{plain}

\title{Thesenpapier – Kant: Grundlegung zur Metaphysik der Sitten}
\subtitle{Abschnitt 1: Übergang zur philosophischen Vernunfterkenntnis (S.\@ 18--26)}
\author{Clemens Damke (Matr.-Nr. 7011488)}
\date{22.04.2019}
\maketitle

\section*{Absatz 1 (18): Die zentrale Rolle des guten Willens}

\begin{itemize}
    \item \textbf{These:} Nur der gute Wille kann für gut gehalten werden.
    \item \textbf{Erörterung:} Naturgaben (z.~B.\@ Verstand, Witz oder Mut) und Glücksgaben (z.~B.\@ Macht, Reichtum oder Gesundheit) sind nicht zwangsläufig gut und können teils sogar bösartig eingesetzt werden.
    \item \textbf{Postulierte Konsequenz:} Der gute Wille ist unerlässliche Bedingung der Würdigkeit glücklich zu sein.
\end{itemize}

\section*{Absatz 2 (18f.): Einordnung hochgeschätzter Eigenschaften}

\begin{itemize}
    \item \textbf{These:} Bestimmte Eigenschaften können den guten Willen zwar befördern, diese haben hierdurch aber keinen inneren unbedingten Wert.
    \item \textbf{Beleg durch Gegenbeispiel:} Die geschätzten Eigenschaften Selbstbeherrschung und nüchterne Überlegtheit können ohne guten Willen kaltblütiges, böses Verhalten bedingen und sind somit nicht inhärent gut.
\end{itemize}

\section*{Absatz 3 (19): Irrelevanz der Durchsetzbarkeit des guten Willens}

\begin{itemize}
    \item \textbf{These:} Der Wert des guten Willens ist unabhängig davon, ob seine Absicht durchgesetzt wird.
		Er ist an sich gut.
    \item \textbf{Konsequenz:} Solange alle verfügbaren Mittel aufgeboten werden um einen guten Willen durchzusetzen, bleibt sein Wert ungemindert;
		selbst wenn dessen Absicht nicht durchsetzbar ist.
\end{itemize}

\section*{Absatz 4 (19f.): Kritik am Wert des bloßen Willens}

\begin{itemize}
    \item \textbf{Kritik:} Die Idee, dass der bloße Wille einen absoluten Wert habe, ist befremdlich.
		Gibt es Anhaltspunkte für diese Idee?
	\item \textbf{Antwort:} Prüfung dieser Idee des absoluten Werts des Willens, indem der Zweck der Vernunftbegabtheit betrachtet wird.
\end{itemize}

\section*{Absatz 5 (20): Zweckmäßigkeit der Vernunft}

\begin{itemize}
    \item \textbf{These:} Die Werkzeuge jedes zweckmäßig zum Leben eingerichteten Wesens sind für seinen Zweck die angemessensten.
		Kant scheint Menschen als derart eingerichtete Wesen aufzufassen.
	\item \textbf{Beobachtung:} Die Vernunft ist nicht das angemessenste Werkzeug für die Glückseligkeit.
		Instinktgeleitetes Handeln ist hierfür, laut Kant, besser geeignet.
\end{itemize}

\section*{Absatz 6 (20f.): Verhältnis zwischen Vernunft und Glückseligkeit}

\begin{itemize}
	\item \textbf{These:} Je mehr sich die Vernunft mit der Absicht auf den Genuss des Lebens und der Glückseligkeit abgibt, desto weiter kommt der Mensch von wahrer Zufriedenheit ab.
	\item \textbf{Konsequenzen:}
		\begin{enumerate}
			\item Entstehung von \textit{Misologie}, d.~h.\@ Hass der Vernunft.
			\item Neid der vernunftgeleiteten nach Glückseligkeit strebenden Menschen auf glückliche instinktgeleitete Menschen.
		\end{enumerate}
	\item \textbf{Konsequenz hieraus:} Die Absicht der Vernunft ist nicht die Glückseligkeit, denn für diese ist sie nicht das angemessenste Werkzeug.
		Ihr muss daher eine andere, laut Kant, viel würdigere Absicht zugrunde liegen.
\end{itemize}

\section*{Absatz 7 (21f.): Der Zweck der Vernunft}

\begin{itemize}
	\item \textbf{These:} Es werden zwei Feststellungen über die Vernunft gemacht:
		\begin{enumerate}
			\item Sie leitet den Willen.
			\item Sie ist untauglich um den Willen zur Befriedigung aller menschlichen Bedürfnisse zu leiten.
			 	Instinkte wären hierfür, wie zuvor diskutiert, geeigneter.
		\end{enumerate}
		Der Zweck der Vernunft kann daher nicht darin liegen, den Willen als Mittel zu einer bestimmten Absicht zu leiten.
		Er liegt laut Kant stattdessen darin, einen an sich guten Willen hervorzubringen.
	\item \textbf{Erörterung:} Kant sieht die Vernunft als das angemessenste Werkzeug zu diesem Zweck:
		\glqq{}\ldots nämlich aus der Erfüllung eines Zwecks, den wiederum nur Vernunft bestimmt, fähig ist\ldots\grqq{} (22).
	\item \textbf{Konsequenz:} Nicht alle Neigungen können erfüllt werden.
\end{itemize}

\end{document}
