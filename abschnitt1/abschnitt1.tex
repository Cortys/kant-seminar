% chktex-file 21
% chktex-file 46
\documentclass{llncs}

\PassOptionsToPackage{utf8}{inputenc}
\usepackage{inputenc}
\usepackage[ngerman]{babel}

% \usepackage[sc]{mathpazo}
\usepackage{geometry}
\geometry{
  a4paper,         % or letterpaper
  textwidth=15cm,  % llncs has 12.2cm
  textheight=23cm, % llncs has 19.3cm
  heightrounded,   % integer number of lines
  hratio=1:1,      % horizontally centered
  vratio=1:1,      % vertically centered
}
\renewcommand{\baselinestretch}{1.15}
\usepackage[parfill]{parskip}

\usepackage{amssymb}
\renewcommand{\labelitemi}{\raisebox{0.09cm}{\tiny\(\blacktriangleright\)}}

\usepackage{graphicx}
\usepackage{enumitem}

\usepackage{hyperref}
\hypersetup{% setup the hyperref-package options
    pdftitle={Kant: Grundlegung zur Metaphysik der Sitten},    %   - title (PDF meta)
    pdfsubject={Thesenpapier},%   - subject (PDF meta)
    pdfauthor={Clemens Damke},    %   - author (PDF meta)
    plainpages=false,           %   -
    colorlinks=false,           %   - colorize links?
    pdfborder={0 0 0},          %   -
    breaklinks=true,            %   - allow line break inside links
    bookmarksnumbered=true,     %
    bookmarksopen=true          %
}

\RequirePackage{lmodern}	% font set: Latin Modern
		\RequirePackage{XCharter}	% font set: Charter
		%\RequirePackage{fourier}	% font set: (basically improved utopia)

		\RequirePackage[sfdefault, light]{roboto}
		\renewcommand*\familydefault{\rmdefault}

\begin{document}
\pagestyle{plain}

\title{Thesenpapier – Kant: Grundlegung zur Metaphysik der Sitten}
\subtitle{Abschnitt 1: Übergang zur philosophischen Vernunfterkenntnis (S.\@ 18--26)}
\author{Clemens Damke (Matr.-Nr. 7011488)}
\date{22.04.2019}
\maketitle

\section*{Absatz 1 (18): Die zentrale Rolle des guten Willens}

\begin{itemize}
    \item \textbf{These:} Nur der gute Wille kann für gut gehalten werden.
    \item \textbf{Erörterung:} Naturgaben (z.~B.\@ Verstand, Witz oder Mut) und Glücksgaben (z.~B.\@ Macht, Reichtum oder Gesundheit) sind nicht zwangsläufig gut und können teils sogar bösartig eingesetzt werden.
    \item \textbf{Postulierte Konsequenz:} Der gute Wille ist unerlässliche Bedingung der Würdigkeit glücklich zu sein.
\end{itemize}

\section*{Absatz 2 (18f.): Einordnung hochgeschätzter Eigenschaften}

\begin{itemize}
    \item \textbf{These:} Bestimmte Eigenschaften können den guten Willen zwar befördern, diese haben hierdurch aber keinen inneren unbedingten Wert.
    \item \textbf{Beleg durch Gegenbeispiel:} Die geschätzten Eigenschaften Selbstbeherrschung und nüchterne Überlegtheit können ohne guten Willen kaltblütiges, böses Verhalten bedingen und sind somit nicht inhärent gut.
\end{itemize}

\section*{Absatz 3 (19): Irrelevanz der Durchsetzbarkeit des guten Willens}

\begin{itemize}
    \item \textbf{These:} Der Wert des guten Willens ist unabhängig davon, ob seine Absicht durchgesetzt wird.
		Er ist an sich gut.
    \item \textbf{Konsequenz:} Solange alle verfügbaren Mittel aufgeboten werden um einen guten Willen durchzusetzen, bleibt sein Wert ungemindert;
		selbst wenn dessen Absicht nicht durchsetzbar ist.
\end{itemize}

\section*{Absatz 4 (19f.): Kritik am Wert des bloßen Willens}

\begin{itemize}
    \item \textbf{Kritik:} Die Idee, dass der bloße Wille einen absoluten Wert habe, ist befremdlich.
		Gibt es Anhaltspunkte für diese Idee?
	\item \textbf{Antwort:} Prüfung der Idee des absoluten Werts des Willens, indem der Zweck der Vernunftbegabtheit betrachtet wird.
\end{itemize}

\section*{Absatz 5 (20): Zweckmäßigkeit der Vernunft}

\begin{itemize}
    \item \textbf{These:} Die Werkzeuge jedes zweckmäßig zum Leben eingerichteten Wesens sind für seinen Zweck die angemessensten.
		Kant scheint Menschen als derart eingerichtete Wesen aufzufassen.
	\item \textbf{Beobachtung:} Die Vernunft ist nicht das angemessenste Werkzeug für die Glückseligkeit.
		Instinktgeleitetes Handeln ist hierfür, laut Kant, besser geeignet.
\end{itemize}

\section*{Absatz 6 (20f.): Verhältnis zwischen Vernunft und Glückseligkeit}

\begin{itemize}
	\item \textbf{These:} Je mehr sich die Vernunft mit der Absicht auf den Genuss des Lebens und der Glückseligkeit abgibt, desto weiter kommt der Mensch von wahrer Zufriedenheit ab.
	\item \textbf{Konsequenzen:}
		\begin{enumerate}
			\item Entstehung von \textit{Misologie}, d.~h.\@ Hass der Vernunft.
			\item Neid der vernunftgeleiteten, nach Glückseligkeit strebenden Menschen auf instinktgeleitete glückliche Menschen.
		\end{enumerate}
	\item \textbf{Konsequenz hieraus:} Die Absicht der Vernunft ist nicht die Glückseligkeit, denn für diese ist sie nicht das angemessenste Werkzeug.
		Ihr muss daher eine andere, laut Kant, viel würdigere Absicht zugrunde liegen.
\end{itemize}

\section*{Absatz 7 (21f.): Der gute Wille als Zweck der Vernunft}

\begin{itemize}
	\item \textbf{These:} Es werden zwei Feststellungen über die Vernunft gemacht:
		\begin{enumerate}
			\item Sie leitet den Willen.
			\item Sie ist untauglich um den Willen zur Befriedigung aller menschlichen Bedürfnisse zu leiten.
			 	Instinkte wären hierfür, wie zuvor diskutiert, geeigneter.
		\end{enumerate}
		Der Zweck der Vernunft kann daher nicht darin liegen, den Willen als Mittel zu einer bestimmten Absicht zu leiten.
		Er liegt laut Kant stattdessen darin, einen an sich guten Willen hervorzubringen.
	\item \textbf{Erörterung:} Kant sieht die Vernunft als das angemessenste Werkzeug zu diesem Zweck:
		\glqq{}\ldots  nämlich aus der Erfüllung eines Zwecks, den wiederum nur Vernunft bestimmt, fähig ist\ldots\grqq{} (22).
	\item \textbf{Konsequenz:} Nicht alle Neigungen können erfüllt werden.
\end{itemize}

\section*{Absatz 8 (22): Verhältnis zwischen gutem Willen und Pflicht}

\begin{itemize}
	\item Kant will nun den Begriff des guten Willens über den Begriff der Pflicht genauer erfassen.
	\item \textbf{Thesen:}
	 	\begin{enumerate}
	 		\item Dem natürlichen gesunden Verstand wohnt der gute Wille bereits bei.
				Der gute Wille muss daher niemandem gelehrt, sondern ``vielmehr nur aufgeklärt'' werden.
			\item Mit dem Begriff der Pflicht kann der gute Wille genauer erfasst, bzw.\@ ``durch Abstechung gehoben'' werden.
	 	\end{enumerate}
\end{itemize}

\section*{Absatz 9 (22f.): Differenzierung zwischen pflichtmäßigem Handeln und Handeln aus Pflicht}

\begin{itemize}
	\item \textbf{Definitionen:}
	 	\begin{enumerate}
	 		\item \textit{Pflichtmäßiges Handeln:} Handeln, das mit der Pflicht im Einklang ist, also nicht pflichtwidrig ist.
				Hat keinen inhärenten moralischen Gehalt.
			\item \textit{Handeln aus Pflicht:} Handeln, das pflichtmäßig ist und zugleich ohne Neigung erfolgt.
				Hat moralischen Gehalt.
	 	\end{enumerate}
	\item \textbf{Problem:} Erkennen, wann aus Pflicht und wann nur pflichtmäßig gehandelt wird.
		\begin{enumerate}
			\item Einfache Fälle: Pflichtwidriges Handeln und pflichtgemäßes Handeln, zu dem Menschen keine unmittelbare Neigung haben.
				Hier lässt sich die Pflichtwidrigkeit bzw.\@ das Handeln aus Pflicht vergleichswiese leicht feststellen.
			\item Schwieriger: Pflichtmäßiges Handeln zu dem eine unmittelbare Neigung des Handelnden besteht.
		\end{enumerate}
	\item \textbf{Beispiel:} Der Kaufmann setzt keinen allgemeinen Preis für jedermann fest, weil es ihm die Pflicht zur Ehrlichkeit gebietet, sondern weil es für ihn von Vorteil ist.
		Seine Ehrlichkeit beruht also auf eigennütziger Absicht und ist somit zwar pflichtgemäß, er handelt aber nicht aus Pflicht.
\end{itemize}

\section*{Absatz 10 (23): Moralischer Gehalt im Erhalt des Lebens}

\begin{itemize}
	\item \textbf{These:} Sein Leben zu erhalten ist Pflicht und für die meisten Menschen zugleich Neigung.
		Dies ist also pflichtgemäß, geschieht aber nicht aus Pflicht und hat somit noch keinen moralischen Gehalt.
	\item \textbf{Ausnahme:} Jemand, der sich den Tod wünscht, und dennoch sein Leben erhält, weil es ihm die Pflicht gebietet, handelt aus Pflicht. Seine \textit{Maxime} hat somit moralischen Gehalt.
	\item \textbf{Begriffsklärung:} Die \textit{Maxime} ist das subjektive Gesetz, nach dem man wirklich handelt (siehe Kritik der praktischen Vernunft).
\end{itemize}

\section*{Absatz 11 (24f.): Moralischer Gehalt der Wohltätigkeit}

\begin{itemize}
	\item \textbf{These:} Wohltätig zu sein ist Pflicht.
		Übliche Beweggründe hierfür sieht Kant in
		\begin{enumerate}
			\item der Eitelkeit oder dem Eigennutz
			\item einem inneren Vergnügen daran, Freude um sich zu verbreiten.
		\end{enumerate}
		In diesen Fällen ist die Wohltätigkeit zwar pflichtgemäß aber geschieht nicht aus Pflicht, da die Beweggründe Neigungen sind.
		Dieses Handeln verdient daher laut Kant Lob, aber keine Hochschätzung, da es keinen sittlichen bzw.\@ moralischen Gehalt hat.
	\item \textbf{Ausnahme:} Ein Mensch ohne Sympathie für Notleidende (aufgrund akuter Umstände oder natürlicher Veranlagung), der dennoch wohltätig ist, handelt hingegen mit sittlichem Gehalt, da er nur aus Pflicht handelt.
\end{itemize}

\section*{Absatz 12 (25): Moralischer Gehalt der eigenen Glückseligkeit}

\begin{itemize}
	\item \textbf{These:} Das Sichern der eigenen Glückseligkeit ist Pflicht. Zugleich hat der Mensch die ``mächtigste und innigste Neigung zur Glückseligkeit''. $\rightarrow$ Kein inhärenter moralischer Gehalt.
	\item \textbf{Begründung dieser Pflicht:} ``Unbefriedigte Bedürfnisse können leicht eine große Versuchung zu Übertretung der Pflichten werden''.
	\item \textbf{Problem:} Nicht alle Neigungen zur Erreichung der Glückseligkeit sind kompatibel.
		Es muss daher eine oftmals uneindeutige Wahl getroffen werden.
		Sofern die Wahl aus Pflicht zur Glückseligkeit, nicht aus Neigung, erfolgt, hat sie einen moralischen Wert;
		unabhängig von der konkret getroffenen Wahl.
\end{itemize}

\section*{Absatz 13 (25f.): Moralischer Gehalt der chistlichen Nächstenliebe}

\begin{itemize}
	\item \textbf{These:} Die biblische Forderung seinen Nächsten, selbst seinen Feind, zu lieben, ist Pflicht.
		Im Überwinden der natürlichen Abneigung dazu sieht Kant eine ``Liebe, die im Willen liegt und nicht im Hange der Empfindung,\,'' und somit moralischen Gehalt hat.
\end{itemize}

\section*{Absatz 14 (26): Einführung des Prinzips des Wollens}

\begin{itemize}
	\item \textbf{These:} Eine Handlung aus Pflicht bezieht ihren moralischen Wert nicht aus ihrer Absicht, sondern aus dem ihr zugrundeliegenden \textit{Prinzips des Wollens}.
	\item \textbf{Begriffsklärung}: Kant sieht den Willen durch zwei Aspekte beeinflusst:
		\begin{enumerate}
			\item \textit{Das Prinzip des Wollens}: Ein formelles Prinzip, welches a priori ist, also unabhängig von materiellen/empirischen Einflüssen und Erkenntnissen.
			\item \textit{Die Triebfeder}: A posteriori, also durch materielle Einflüsse bestimmt.
		\end{enumerate}
	\item \textbf{Begründung der These:} Die Beispiele aus den vorigen Absätzen illustrieren, dass zwei Handlungen mit derselben angestrebten Wirkung einen unterschiedlichen moralischen Wert haben können (pflichtgemäßes Handeln vs.\@ Handeln aus Pflicht).
		Die Quelle des moralischen Werts sieht Kant daher im Prinzip des Wollens, denn materielle Prinzipien (die Triebfeder des Wollens) sollen keinen Einfluss haben.
\end{itemize}

\end{document}
